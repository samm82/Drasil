\documentclass[12pt]{article}
\usepackage{fontspec}
\usepackage{fullpage}
\usepackage{hyperref}
\hypersetup{bookmarks=true,colorlinks=true,linkcolor=red,citecolor=blue,filecolor=magenta,urlcolor=cyan}
\usepackage{amsmath}
\usepackage{amssymb}
\usepackage{mathtools}
\usepackage{unicode-math}
\usepackage{tabu}
\usepackage{longtable}
\usepackage{booktabs}
\usepackage{caption}
\usepackage{enumitem}
\usepackage{filecontents}
\usepackage[backend=bibtex]{biblatex}
\usepackage{url}
\setmathfont{Latin Modern Math}
\newcommand{\gt}{\ensuremath >}
\newcommand{\lt}{\ensuremath <}
\global\tabulinesep=1mm
\bibliography{bibfile}
\title{Software Requirements Specification for Chemistry Code}
\author{Samuel J. Crawford}
\begin{document}
\maketitle
\tableofcontents
\newpage
\section{Reference Material}
\label{Sec:RefMat}
This section records information for easy reference.

\subsection{Table of Symbols}
\label{Sec:ToS}
The symbols used in this document are summarized in the \hyperref[Table:ToS]{Table of Symbols} along with their units. The symbols are listed in alphabetical order.

\begin{longtable}{l l l}
\toprule
\textbf{Symbol} & \textbf{Description} & \textbf{Units}
\\
\midrule
\endhead
$r$ & Representation of a chemical equation & --
\\
\bottomrule
\caption{Table of Symbols}
\label{Table:ToS}
\end{longtable}
\subsection{Abbreviations and Acronyms}
\label{Sec:TAbbAcc}
\begin{longtable}{l l}
\toprule
\textbf{Abbreviation} & \textbf{Full Form}
\\
\midrule
\endhead
ChemCode & Chemistry Code
\\
\bottomrule
\caption{Abbreviations and Acronyms}
\label{Table:TAbbAcc}
\end{longtable}
\section{Introduction}
\label{Sec:Intro}
Chemical equations are common ways of representing chemical reactions but they must be balanced \cite{lund2023}. This process of balancing a chemical equation involves introducing coefficients before each chemical formula such that there are the same number of atoms of each element on the reactant and product sides of the chemical equation. The program documented here is called Chemistry Code (ChemCode).

The following section provides an overview of the Software Requirements Specification (SRS) for ChemCode. This section explains the purpose of this document, the scope of the requirements, the characteristics of the intended reader, and the organization of the document.

\subsection{Scope of Requirements}
\label{Sec:ReqsScope}
The scope of the requirements includes all chemical equations with at most one more compound than element.

\section{Requirements}
\label{Sec:Requirements}
This section provides the functional requirements, the tasks and behaviours that the software is expected to complete, and the non-functional requirements, the qualities that the software is expected to exhibit.

\subsection{Functional Requirements}
\label{Sec:FRs}
This section provides the functional requirements, the tasks and behaviours that the software is expected to complete.

\begin{itemize}
\item[Input-Values:\phantomsection\label{inputValues}]{Input the values from \hyperref[Table:ReqInputs]{Tab:ReqInputs}.}
\item[Convert-to-Matrix:\phantomsection\label{convertMatrix}]{Convert the inputted chemical equation to matrix form.}
\end{itemize}
\begin{longtable}{l l l}
\toprule
\textbf{Symbol} & \textbf{Description} & \textbf{Units}
\\
\midrule
\endhead
$r$ & Representation of a chemical equation & --
\\
\bottomrule
\caption{Required Inputs following \hyperref[inputValues]{FR:Input-Values}}
\label{Table:ReqInputs}
\end{longtable}
\subsection{Non-Functional Requirements}
\label{Sec:NFRs}
This section provides the non-functional requirements, the qualities that the software is expected to exhibit.

\begin{itemize}
\item[Accurate:\phantomsection\label{accurate}]{Chemical equations are only useful if they are balanced, so computed coefficients should be exact.}
\item[Verifiable:\phantomsection\label{verifiable}]{The code is tested with complete verification and validation plan.}
\item[Reusable:\phantomsection\label{reusable}]{The code is modularized.}
\item[Portable:\phantomsection\label{portable}]{The code is able to be run in different environments.}
\end{itemize}
\section{References}
\label{Sec:References}
\begin{filecontents*}{bibfile.bib}
@book{lund2023,
author={Lund, Lance},
title={Introduction to Chemistry},
publisher={LibreTexts},
year={2023},
address={Cambridge and Coon Rapids, MN, USA},
month=jan,
howpublished={\url{https://chem.libretexts.org/Courses/Anoka-Ramsey\_Community\_College/Introduction\_to\_Chemistry}}}
\end{filecontents*}
\nocite{*}
\bibstyle{ieeetr}
\printbibliography[heading=none]
\end{document}
