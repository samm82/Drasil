\documentclass[12pt, titlepage]{article}

\usepackage{booktabs}
\usepackage{tabularx}
\usepackage{colortbl}
\usepackage{xltabular}
\usepackage{xr-hyper}
\usepackage{hyperref}
\hypersetup{
    colorlinks,
    citecolor=blue,
    filecolor=black,
    linkcolor=red,
    urlcolor=blue
}
\usepackage[version=4]{mhchem}
\usepackage{pdflscape}

\makeatletter
\newcommand*{\addFileDependency}[1]{% argument=file name and extension
\typeout{(#1)}% latexmk will find this if $recorder=0
% however, in that case, it will ignore #1 if it is a .aux or 
% .pdf file etc and it exists! If it doesn't exist, it will appear 
% in the list of dependents regardless)
%
% Write the following if you want it to appear in \listfiles 
% --- although not really necessary and latexmk doesn't use this
%
\@addtofilelist{#1}
%
% latexmk will find this message if #1 doesn't exist (yet)
\IfFileExists{#1}{}{\typeout{No file #1.}}
}\makeatother

\newcommand*{\myexternaldocument}[1]{%
\externaldocument{#1}%
\addFileDependency{#1.tex}%
\addFileDependency{#1.aux}%
}

\newcounter{testnum} %NFR Number
\newcommand{\tthetestnum}{NFR\thetestnum}
\newcommand{\testref}[1]{T\ref{#1}}

\newcommand{\dref}[1]{GD\ref{#1}}
\newcommand{\ddref}[1]{DD\ref{#1}}
\newcommand{\tmref}[1]{TM\ref{#1}}
\newcommand{\tbref}[1]{TB\ref{#1}}
\newcommand{\aref}[1]{A\ref{#1}}
\newcommand{\gsref}[1]{GS\ref{#1}}
\newcommand{\iref}[1]{IM\ref{#1}}
\newcommand{\rref}[1]{R\ref{#1}}
\newcommand{\nfrref}[1]{NFR\ref{#1}}
\newcommand{\lcref}[1]{LC\ref{#1}}
\newcommand{\ucref}[1]{UC\ref{#1}}

\input{../Comments}
\input{../Common}

\myexternaldocument{../SRS/SRS}

\begin{document}

\title{%Project Title: 
  System Verification and Validation Plan for \progname{}}
\author{\authname}
\date{\today}

\maketitle

\pagenumbering{roman}

\tableofcontents

\listoftables

% \listoffigures
% \wss{Remove this section if it isn't needed}

\newpage

\section{Revision History}

\begin{xltabular}{\textwidth}{llX}

  \toprule {\bf Date} & {\bf Version} & {\bf Notes}                          \\
  \midrule
  \endhead

  \bottomrule
  \endfoot

  Feb. 5, 2023        & 0.0           & Create document and remove
  inapplicable content                                                       \\
  Feb. 7-8, 2023      & 0.1.0         & Add input tests                      \\
  Feb. 8, 2023        & 0.1.1         & Improve referencing of tests         \\
  & 0.1.2         & Add matrix conversion tests and
  improve input tests, including rationale, labelling, and chemical
  equations                                                                  \\
  Feb. 9, 2023        & 0.1.3         & Add tests for trivial equation       \\
  & 0.1.4         & Add feasibility tests                \\
  Feb. 13, 2023       & 0.1.5         & Clarify notion of matrices having
  the same solution after swapping rows and/or columns                       \\
  Feb. 13-14, 2023    & 0.1.6         & Add balancing and output tests       \\
  Feb. 14, 2023       & 0.1.7         & Add accuracy test for balancing      \\
  Feb. 16, 2023       & 0.1.8         & Move (reordered) ``system tests'' to
  unit test section (commented out), convert them to true system tests, and
  tweak balancing accuracy test appropriately                                \\
  & 0.1.9         & Add static test for element support  \\
  Feb. 18, 2023       & 0.1.10        & Improve formatting of existing
  tests, including adding references to requirements                         \\
  Feb. 19, 2023       & 0.2.0         & Add remaining tests for
  nonfunctional requirements (except for \nfrref{NFR_maintainability})       \\
  & 0.2.1         & Fill in \nameref{sec_plan} section   \\
  & 0.3.0         & Fill in \nameref{sec_gen_info}
  section                                                                    \\
  Feb. 20, 2023       & 0.3.1         & Add references to requirements and
  external documents                                                         \\
  & 0.3.2         & Finish \nameref{sec_abbrsAcrs}
  section                                                                    \\
  & 0.3.3         & Fill in \nameref{sec_unit_tests}
  section and remove its template content                                    \\
  & 0.3.4         & Add final pieces of information and
  improve formatting                                                         \\
  & 1.0           & Add traceability matrix in Section
  \ref{sec_test_req_trace}                                                   \\
  Mar. 4, 2023 & 1.1.0 & Replace method of solving with integer programming and
  improve Revision History table format \\
\end{xltabular}

\section{Abbreviations and Acronyms} \label{sec_abbrsAcrs}

In addition to the ones from the \nameref{srs_sec_abbsAcrs} section from the
SRS \cite{srs}, the following abbreviations and acronyms are used throughout
this document:

~\newline\noindent
\renewcommand{\arraystretch}{1.2}
\begin{tabular}{l l}
  \toprule
  \textbf{symbol} & \textbf{description}   \\
  \midrule
  CAS             & Computing and Software \\
  SUS             & System Usability Scale \\
  T               & Test                   \\
  \bottomrule
\end{tabular}\\

\newpage

\pagenumbering{arabic}

\section{General Information} \label{sec_gen_info}

This document outlines the plan for verifying and validating \progname{}. It
provides a brief overview of \progname{}, as well as the main objectives of
this document and references to other relevant documentation, an overview of
the verification and validation of various artifacts, a description of the
system and unit tests, and any other relevant information.

\subsection{Summary}

The software that will be tested is \progname{}, a tool for automatically
balancing chemical equations so that they may be useful
\cite{lund_introduction_2023}.

\subsection{Objectives}

The primary objectives of this plan are to build confidence in the software's
correctness and to ensure that this project meets the goals of both CAS 741 and
Drasil.

\subsection{Relevant Documentation}

This plan outlines the process for verifying the Software Requirements
Specification (SRS) \cite{srs} and this Verification and Validation
(VnV) Plan in Sections \ref{sec_srs_verif_plan} and \ref{sec_vnv_verif_plan},
respectively; these are the only relevant pieces of documentation for this
plan. (Note that design documentation would also be relevant, but since
\progname{} is being built in Drasil which doesn't generate this documentation
\cite{carette_drasil_2021}, it is out of the scope of this project.)

\section{Plan} \label{sec_plan}

This section outlines the plan for verifying and validating \progname{}, which
will be performed by a team. This plan includes verifying the SRS,
design, VnV Plan, and implementation of \progname{}. It also includes the use
of automated tools for testing and verification and external data for
validation.

\subsection{Verification and Validation Team} \label{sec_vnv_team}

Each teammate has their role(s) listed in the \nameref{table_team} with the
roles defined as follows:

\begin{itemize}
  \item \textbf{Author}: The person writing the documentation and Drasil
        implementation of \progname{}.

  \item \textbf{Project Supervisor}: The person in charge of the \progname{}
        project, including its implementation in Drasil.

  \item \textbf{Reviewer}: A person in charge of ensuring that the
        documentation and/or implementation of \progname{} meets the needs of
        the CAS 741 course (this may be limited to one artifact).

  \item \textbf{Validator}: A person in charge of ensuring that the
        documentation and/or implementation of \progname{} meets the goals of
        its implementation in Drasil.

  \item \textbf{Verifier}: A person in charge of ensuring that the
        implementation of \progname{} meets the requirements from its SRS.

\end{itemize}

\begin{table}[h!]
  \centering
  \begin{tabular}{| l | l |}
    \hline
    \rowcolor[gray]{0.9}
    \bf Name          & \bf Role                                \\
    \hline
    Dr.~Spencer Smith & Project Supervisor, Reviewer, Validator \\
    \hline
    Samuel Crawford   & Author, Reviewer, Validator, Verifier   \\
    \hline
    Jason Balaci      & Reviewer, Validator                     \\
    \hline
    Deesha Patel      & SRS Reviewer                            \\
    \hline
    Maryam Valian     & VnV Plan Reviewer                       \\
    \hline
    Karen Wang        & Drasil Implementation Reviewer          \\
    \hline
    Drasil Team       & Validators, Verifiers                   \\
    \hline
    Class of CAS 741  & Reviewers                               \\
    \hline
  \end{tabular}
  \caption{Table of Teammates and Their Roles}
  \label{table_team}
\end{table}

\subsection{SRS Verification Plan} \label{sec_srs_verif_plan}

The first round of review for the SRS will be done on its manual version.
It will be reviewed first by Samuel Crawford during the writing process, then
generally by the class of CAS 741, and then more rigorously by Jason Balaci,
Deesha Patel, and Dr.~Spencer Smith. After appropriate revisions and its
implementation in Drasil, the generated version will then be reviewed by Samuel
Crawford, Dr.~Spencer Smith and a subset of the Drasil team. Each review will
check it against the general writing checklist \cite{writing_checklist}, the
SRS checklist \cite{srs_checklist} and the SRS rubric for CAS 741 \sjc{I
  couldn't find a link for this.}; the only exception is that the Drasil team
will only verify the SRS within the context of Drasil (so they will not check
it against the CAS 741 rubric, for example).

\subsection{Design Verification Plan} \label{sec_desVerPlan}

The design of \progname{} will be dictated by the code generation of Drasil
\cite{carette_drasil_2021}. As Drasil does not currently generate design
documentation \cite{carette_drasil_2021} and writing it is outside the scope of
the \progname{} project, this verification will be conducted
based on the generated code itself. Verifying \progname's
design will be done by the Drasil team according to their process of design
verification and is therefore out of the scope of this document.

\subsection{Verification and Validation Plan Verification Plan}
\label{sec_vnv_verif_plan}

The VnV Plan will be reviewed first by Samuel Crawford during the writing
process, then generally by the class of CAS 741, and then more rigorously by
Jason Balaci, Maryam Valian, and Dr.~Spencer Smith. After appropriate
revisions, it will then be reviewed again by Samuel Crawford and Dr.~Spencer
Smith. Each review will check it against the general writing checklist
\cite{writing_checklist}, the VnV checklist \cite{vnv_checklist},
and the VnV rubric for CAS 741 \sjc{I couldn't find a link for this.}.

\subsection{Implementation Verification Plan}

The primary method of verifying the implementation of \progname{} will be
performing the system and unit tests from Sections \ref{sec_sys_tests} and
\ref{sec_unit_tests}, respectively. Since the implementation will be
generated by Drasil, a subset of the Drasil team will also inspect the code to
ensure that it matches the expectations for generated code, especially focusing
on the new code generation for solving integer programming problems.
The implementation will also likely be reviewed by Drasil team members in the
future, as related to future work on Drasil.

\subsection{Automated Testing and Verification Tools}

Since the initial implementation of \progname{} will be in Python, pytest will
be used to automate testing and measure coverage where appropriate. Other
frameworks for automated testing will be added if/when code for \progname{} is
generated in more languages. The Drasil repository uses continuous integration
to ensure the code follows HLint standards and builds properly and that the
generated artifacts match the ``stable'' (i.e., manually verified) versions.
Since Drasil also generates Makefiles \cite{carette_drasil_2021}, they will
also be used. Since the code for \progname{} will be generated
by Drasil, the Drasil team may potentially use other tools as part of its
verification (e.g., linters).

\subsection{Software Validation Plan}

There are two sets of goals for \progname{}: those for CAS 741 and those for
Drasil. The verification of the SRS from Section \ref{sec_srs_verif_plan}
will help ensure that \progname{} satisfies both sets of goals. External
chemical reaction data, such as that from the NIST Chemical Kinetics Database
\cite{national_institute_of_standards_and_technology_nist_2023}, may also be
used to validate this software.

\newpage

\section{System Test Description} \label{sec_sys_tests}

\subsection{Tests for Functional Requirements} \label{sec_sysFunReqs}

The tests in each section are given in order of increasing
complexity/likelihood of the situation arising during use of \progname{}.

\subsubsection{Element Support Testing}

\begin{enumerate}

  \item[T\refstepcounter{testnum}\thetestnum \label{test_element_support}:]
    \textbf{Test for Element Support}

    Test Case Derivation: In order for the user to be able to work with any
    possible chemical reaction, they must be able to enter each of the
    currently known 118 elements.

    How test will be performed: Static analysis will be performed against the
    list of elements from \cite{wikipedia_list_2023}, ensuring that the
    symbol of every element can be used throughout the execution of
    \progname{}. This includes the input stage (from \rref{R_input}), the
    conversion stage (from \rref{R_convert}), and the output stage (from
    \rref{R_infeasOutput} and \rref{R_feasOutput} from the SRS \cite{srs}).

\end{enumerate}

\subsubsection{Feasible Reaction Testing}

The following tests are for equations of feasible chemical reactions and
will be performed automatically. Providing the correct output for an inputted
feasible chemical reaction satisfies \rref{R_input}, \rref{R_convert},
\rref{R_feasible}, \rref{R_balance}, and \rref{R_feasOutput} from the SRS
\cite{srs}. Note that the input/output format
for each test is an abstract representation, as the specific format of each
input/output is a design decision that is not made at this stage.

\begin{enumerate}

  \item[T\refstepcounter{testnum}\thetestnum \label{test_small_valid_eqn}:]
    \textbf{Test for Small Valid Equation}

    Input: $\ce{O2} \rightarrow \ce{O3}$ \cite{fahey_twenty_2011}

    Output: $\ce{3O2} \rightarrow \ce{2O3}$ \cite[p.~6]{fahey_twenty_2011}

    Test Case Derivation: The inputted chemical equation is
    valid and trivial.

    \newpage

  \item[T\refstepcounter{testnum}\thetestnum \label{test_valid_eqn}:]
    \textbf{Test for Valid Equation}

    Input: $\ce{C2H6} + \ce{O2} \rightarrow \ce{CO2} + \ce{H2O}$
    \cite{hamid_balancing_2019}

    Output: $\ce{2C2H6} + \ce{7O2} \rightarrow \ce{4CO2} + \ce{6H2O}$
    \cite[p.~523]{hamid_balancing_2019}

    Test Case Derivation: The inputted chemical equation is valid and
    relatively small, but larger than the trivial one from
    \testref{test_small_valid_eqn}.

  \item[T\refstepcounter{testnum}\thetestnum \label{test_large_valid_eqn}:]
    \textbf{Test for Large Valid Equation}

    Input: $\ce{KMnO4} + \ce{HCl} \rightarrow \ce{MnCl2} + \ce{KCl} +
      \ce{Cl2} + \ce{H2O}$ \cite{taylor_balancing_2021}

    Output: $\ce{2KMnO4} + \ce{16HCl} \rightarrow \ce{2MnCl2} + \ce{2KCl} +
      \ce{5Cl2} + \ce{8H2O}$ \cite{taylor_balancing_2021}

    Test Case Derivation: The inputted chemical equation is
    valid and larger than the one from \testref{test_valid_eqn}.

  \item[T\refstepcounter{testnum}\thetestnum \label{test_nonstoich_valid_eqn}:]
    \textbf{Test for Valid Equation with Nonstoichiometric Compound}

    Input: $\ce{Fe_{0.95}O} + \ce{O2} \rightarrow \ce{Fe2O3}$
    \cite{doubtnut_when_nodate}

    Output: $\ce{80Fe_{0.95}O} + \ce{17O2} \rightarrow \ce{38Fe2O3}$

    Test Case Derivation: The inputted chemical equation contains a
    nonstoichiometric compound (i.e., one with a fractional subscript).

\end{enumerate}

\subsubsection{Infeasible Reaction Testing}

The following tests are for equations of infeasible chemical reactions and
will be performed automatically. Providing the correct output for an inputted
infeasible chemical reaction satisfies \rref{R_input}, \rref{R_convert},
\rref{R_feasible}, and \rref{R_infeasOutput} from the SRS \cite{srs}.
Note that the input/output format
for each test is an abstract representation, as the specific format of each
input/output is a design decision that is not made at this stage.

\begin{enumerate}

  \item[T\refstepcounter{testnum}\thetestnum \label{test_inf_cons_mass_valid_eqn}:]
    \textbf{Test for Valid Equation that is Infeasible due to Conservation of Mass
      Violation}

    Input: $\ce{C2H6} \rightarrow \ce{CO2} + \ce{H2O}$

    Output: ``This reaction is infeasible because \ce{O} is present on one side
    of the equation but not the other, violating the Law of Conservation of
    Mass.''

    Test Case Derivation: The inputted chemical equation is infeasible since
    every element does not exist on both sides of the equation, which violates
    the Law of Conservation of Mass (\tmref{TM_ConsMass} from the SRS
    \cite{srs}).

  \item[T\refstepcounter{testnum}\thetestnum \label{test_inf_over_valid_eqn}:]
    \textbf{Test for Valid Equation that is Infeasible due to Overconstrained
      System}

    Input: $\ce{K4FeC6N6} + \ce{K2S2O3} \rightarrow \ce{CO2} + \ce{K2SO4} +
      \ce{NO2} + \ce{FeS}$ (modified from \cite{hamid_balancing_2019})

    Output: ``This reaction is infeasible because it is overconstrained.''

    Test Case Derivation: The inputted chemical equation is infeasible since
    each compound has more than one element, so changing any coefficient
    affects the number of some other element, causing a chain reaction that
    does not converge. There is no solution to this system other than the
    trivial solution ($\mathbf{0}$) \cite{hamid_balancing_2019}.

\end{enumerate}

\subsection{Tests for Nonfunctional Requirements}

There are no tests for maintainability because this is the responsibility of
the Drasil team, as Drasil is responsible for generated the actual
implementation of \progname{}.

\subsubsection{Accuracy Testing}

\begin{enumerate}

  \item[T\refstepcounter{testnum}\thetestnum \label{test_bal_accuracy}:]
    \textbf{Test for Accuracy of Balancing}

    Type: Dynamic, Automatic

    Test Case Derivation: Chemical equations are only useful if they are
    balanced \cite{lund_introduction_2023}, so computed coefficients should be
    exact. Since these coefficients should be the smallest possible whole
    numbers \cite{lund_introduction_2023}, there is exactly one possible set of
    coefficients for each feasible chemical equation and exact equality between
    this expected output and the actual output can be checked by a computer
    (e.g., by using integer comparison).

    How test will be performed: This verification will be done by performing
    the \nameref{sec_sysFunReqs} using integer comparison and satisfies
    \rref{R_balance} and \nfrref{NFR_accuracy} from the SRS \cite{srs}.

\end{enumerate}

\newpage

\subsubsection{Understandability Testing}

\begin{enumerate}

  \item[T\refstepcounter{testnum}\thetestnum \label{test_understand}:]
    \textbf{Test for Understandability of \progname{}}

    Type: Dynamic, Manual

    How test will be performed: New typical users will be asked to input a
    valid chemical reaction to be balanced by the system. At least
    USER\_FRAC of users should be able to do so within
    FIRST\_USE\_TIME. This test satisfies \nfrref{NFR_understandability} from
    the SRS \cite{srs}.

\end{enumerate}

\subsubsection{Usability Testing}

\begin{enumerate}

  \item[T\refstepcounter{testnum}\thetestnum \label{test_usable}:]
    \textbf{Test for Usability of \progname{}}

    Type: Dynamic, Manual

    How test will be performed: Typical users will be told how \progname{}
    works and then instructed to input the chemical equations from the
    \nameref{sec_sysFunReqs} to be balanced. They will then be asked the
    questions from the System Usability Scale (SUS) \cite{thomas_how_2015}
    (see \nameref{sec_usableSurvey}). The average score should be at least
    SUS\_SCORE. This test satisfies \nfrref{NFR_usability} from
    the SRS \cite{srs}.

\end{enumerate}

\subsubsection{Portability Testing}

\begin{enumerate}

  \item[T\refstepcounter{testnum}\thetestnum \label{test_portable}:]
    \textbf{Test for Portability of \progname{}}

    Type: Dynamic

    Test Case Derivation: \progname{} may be used by a large variety of users
    who may have different systems.

    How test will be performed: This verification will be performed by
    ensuring the correct programming language(s) is/are installed on both a
    Windows and macOS system and the \nameref{sec_sysFunReqs} will be
    performed. This test satisfies \nfrref{NFR_portability} from the SRS
    \cite{srs}.

\end{enumerate}

\subsubsection{Verifiability Testing}

\begin{enumerate}

  \item[T\refstepcounter{testnum}\thetestnum \label{test_verifiable}:]
    \textbf{Test for Verifiability of \progname{}}

    Type: Dynamic, Static

    How test will be performed: All other tests from this plan will be
    performed successfully. This test satisfies \nfrref{NFR_verifiability}
    from the SRS \cite{srs}.

\end{enumerate}

\begin{landscape}

  \subsection{Traceability Between Test Cases and Requirements} \label{sec_test_req_trace}

  The purpose of the traceability matrices is to provide easy references on what
  has to be additionally modified if a certain component is changed.  Every time a
  component is changed, the items in the column of that component that are marked
  with an ``X'' may have to be modified as well. Table~\ref{Table:trace} shows the
  dependencies of tests on the requirements from the SRS \cite{srs}.

  \begin{table}[h!]
    \centering
    \begin{tabular}{|c|c|c|c|c|c|c|c|c|c|c|c|c|}
      \hline
                                             & \rref{R_input} & \rref{R_convert} & \rref{R_feasible} & \rref{R_infeasOutput} & \rref{R_balance} & \rref{R_feasOutput} & \nfrref{NFR_accuracy} & \nfrref{NFR_understandability} & \nfrref{NFR_usability} & \nfrref{NFR_maintainability} & \nfrref{NFR_portability} & \nfrref{NFR_verifiability} \\ \hline
      \testref{test_element_support}         & X              & X                &                   & X                     &                  & X                   &                       &                                &                        &                              &                          &                            \\ \hline
      \testref{test_small_valid_eqn}         & X              & X                & X                 &                       & X                & X                   &                       &                                &                        &                              &                          &                            \\ \hline
      \testref{test_valid_eqn}               & X              & X                & X                 &                       & X                & X                   &                       &                                &                        &                              &                          &                            \\ \hline
      \testref{test_large_valid_eqn}         & X              & X                & X                 &                       & X                & X                   &                       &                                &                        &                              &                          &                            \\ \hline
      \testref{test_nonstoich_valid_eqn}     & X              & X                & X                 &                       & X                & X                   &                       &                                &                        &                              &                          &                            \\ \hline
      \testref{test_inf_cons_mass_valid_eqn} & X              & X                & X                 & X                     &                  &                     &                       &                                &                        &                              &                          &                            \\ \hline
      \testref{test_inf_over_valid_eqn}      & X              & X                & X                 & X                     &                  &                     &                       &                                &                        &                              &                          &                            \\ \hline
      \testref{test_bal_accuracy}            &                &                  &                   &                       & X                &                     & X                     &                                &                        &                              &                          &                            \\ \hline
      \testref{test_understand}              &                &                  &                   &                       &                  &                     &                       & X                              &                        &                              &                          &                            \\ \hline
      \testref{test_usable}                  &                &                  &                   &                       &                  &                     &                       &                                & X                      &                              &                          &                            \\ \hline
      \testref{test_portable}                &                &                  &                   &                       &                  &                     &                       &                                &                        &                              & X                        &                            \\ \hline
      \testref{test_verifiable}              &                &                  &                   &                       &                  &                     &                       &                                &                        &                              &                          & X                          \\ \hline
    \end{tabular}
    \caption{Traceability Matrix Showing the Connections Between Tests and Requirements}
    \label{Table:trace}
  \end{table}

\end{landscape}

\section{Unit Test Description} \label{sec_unit_tests}

As mentioned in the \nameref{sec_desVerPlan}, design documentation will not be
written manually or generated by Drasil. Therefore, any unit tests will be
based on the generated code itself. Unit testing may be outside the scope of
the \progname{} project \sjc{Is this true?}, but if not, this section will be
filled in once the code for \progname{} is generated by Drasil.

% \subsection{Unit Testing Scope}

% \wss{What modules are outside of the scope.  If there are modules that are
%   developed by someone else, then you would say here if you aren't planning on
%   verifying them.  There may also be modules that are part of your software, but
%   have a lower priority for verification than others.  If this is the case,
%   explain your rationale for the ranking of module importance.}

% \subsection{Tests for Functional Requirements}

% \wss{Most of the verification will be through automated unit testing.  If
%   appropriate specific modules can be verified by a non-testing based
%   technique.  That can also be documented in this section.}

% \subsubsection{Input Testing}

% In order for \progname{} to be useful, it needs to be able to receive a
% chemical equation from the user and store it to balance it later. This
% section defines tests for inputting chemical equations from R1 of the SRS
% \sjc{Add link}. \sjc{Justify the choice of these specific tests.}

% \begin{enumerate}

%   \item[T\refstepcounter{testnum}\thetestnum \label{test_small_valid_input}:]
%     \textbf{Test for Small Valid Input}

%     Control: Manual

%     Initial State: \progname{} is started.

%     Input: A representation of the equation $\ce{O2} \rightarrow \ce{O3}$
%     \cite{fahey_twenty_2011}.

%     Output: The inputted chemical equation is stored in \progname{}.

%     Test Case Derivation: The inputted chemical equation is
%     valid and trivial.

%     How test will be performed: A debug statement will be added to display the
%     stored chemical equation and this representation will be manually compared to
%     the given input.

%   \item[T\refstepcounter{testnum}\thetestnum \label{test_valid_input}:]
%     \textbf{Test for Valid Input}

%     Control: Manual

%     Initial State: \progname{} is started.

%     Input: A representation of the equation
%     $\ce{C2H6} + \ce{O2} \rightarrow \ce{CO2} + \ce{H2O}$
%     \cite{hamid_balancing_2019}.

%     Output: The inputted chemical equation is stored in \progname{}.

%     Test Case Derivation: The inputted chemical equation is valid and
%     relatively small, but larger than the trivial one from
%     \testref{test_small_valid_input}.

%     How test will be performed: A debug statement will be added to display the
%     stored chemical equation and this representation will be manually compared to
%     the given input.

%   \item[T\refstepcounter{testnum}\thetestnum \label{test_large_valid_input}:]
%     \textbf{Test for Large Valid Input}

%     Control: Manual

%     Initial State: \progname{} is started.

%     Input: A representation of the following equation from
%     \cite{taylor_balancing_2021}.
%     $$\ce{KMnO4} + \ce{HCl} \rightarrow \ce{MnCl2} + \ce{KCl} + \ce{Cl2} +
%       \ce{H2O}$$

%     Output: The inputted chemical equation is stored in \progname{}.

%     Test Case Derivation: The inputted chemical equation is
%     valid and larger than the one from \testref{test_valid_input}.

%     How test will be performed: A debug statement will be added to display the
%     stored chemical equation and this representation will be manually compared to
%     the given input.

%   \item[T\refstepcounter{testnum}\thetestnum \label{test_inf_cons_mass_valid_input}:]
%     \textbf{Test for Valid Input that is Infeasible due to Conservation of Mass
%       Violation}

%     Control: Manual

%     Initial State: \progname{} is started.

%     Input: A representation of the equation
%     $\ce{C2H6} \rightarrow \ce{CO2} + \ce{H2O}$.

%     Output: The inputted chemical equation is stored in \progname{}.

%     Test Case Derivation: The inputted chemical equation is infeasible since
%     every element does not exist on both sides of the equation, which violates
%     the Law of Conservation of Mass (\tmref{TM_ConsMass} from the SRS
%     \cite{srs}).

%     How test will be performed: A debug statement will be added to display the
%     stored chemical equation and this representation will be manually compared to
%     the given input.

%   \item[T\refstepcounter{testnum}\thetestnum \label{test_inf_over_valid_input}:]
%     \textbf{Test for Valid Input that is Infeasible due to Overconstrained
%       System}

%     Control: Manual

%     Initial State: \progname{} is started.

%     Input: A representation of the following equation (modified from
%     \cite{hamid_balancing_2019}).
%     $$\ce{K4FeC6N6} + \ce{K2S2O3} \rightarrow \ce{CO2} + \ce{K2SO4} + \ce{NO2} +
%       \ce{FeS}$$

%     Output: The inputted chemical equation is stored in \progname{}.

%     Test Case Derivation: The inputted chemical equation is infeasible since
%     each compound has more than one element, so changing any coefficient
%     affects the number of some other element, causing a chain reaction that
%     does not converge. There is no solution to this system other than the
%     trivial solution ($\mathbf{0}$) \cite{hamid_balancing_2019}.

%     How test will be performed: A debug statement will be added to display the
%     stored chemical equation and this representation will be manually compared to
%     the given input.

%   \item[T\refstepcounter{testnum}\thetestnum \label{test_nonstoich_valid_input}:]
%     \textbf{Test for Valid Input with Nonstoichiometric Compound}

%     Control: Manual

%     Initial State: \progname{} is started.

%     Input: A representation of the equation
%     $\ce{Fe_{0.95}O} + \ce{O2} \rightarrow \ce{Fe2O3}$
%     \cite{doubtnut_when_nodate}.

%     Output: The inputted chemical equation is stored in \progname{}.

%     Test Case Derivation: The inputted chemical equation contains a
%     nonstoichiometric compound (i.e., one with a fractional subscript).

%     How test will be performed: A debug statement will be added to display the
%     stored chemical equation and this representation will be manually compared to
%     the given input.

% \end{enumerate}

% \subsubsection{Matrix Conversion Testing}

% To solve a system of linear equations, \progname{} must first convert the
% inputted chemical equation into matrix form. This section defines tests for
% the matrix form conversion of chemical equations' stored representations from
% R2 of the SRS \sjc{Add link}. \sjc{Justify the choice of these specific tests.}
% Note that since swapping rows and/or columns in a matrix doesn't change its
% solution, the matrices outputted may have their rows and/or columns in a
% different order.

% \begin{enumerate}

%   \item[T\refstepcounter{testnum}\thetestnum \label{test_convert_small_valid}:]
%     \textbf{Test for Converting Small Valid Chemical Equation}

%     Control: Automated

%     Initial State: \progname{} is started with a representation of the equation
%     $\ce{O2} \rightarrow \ce{O3}$ \cite{fahey_twenty_2011} stored in memory.

%     Input: --

%     Output: The matrix
%     $\begin{bmatrix}
%         2 & -3
%       \end{bmatrix}$ or
%     $\begin{bmatrix}
%         -3 & 2
%       \end{bmatrix}$.

%     Test Case Derivation: The stored chemical equation is valid and trivial.

%     How test will be performed: The outputted matrix form of the stored
%     chemical equation will be automatically compared to the expected output.

%   \item[T\refstepcounter{testnum}\thetestnum \label{test_convert_valid}:]
%     \textbf{Test for Converting Valid Chemical Equation}

%     Control: Automated

%     Initial State: \progname{} is started with a representation of the equation
%     $\ce{C2H6} + \ce{O2} \rightarrow \ce{CO2} + \ce{H2O}$
%     \cite{hamid_balancing_2019} stored in memory.

%     Input: --

%     Output: Some matrix with the rows and columns of the following matrix from
%     \cite{hamid_balancing_2019}, potentially in a different order:
%     $$\begin{bmatrix}
%         2 & 0 & -1 & 0  \\
%         6 & 0 & 0  & -2 \\
%         0 & 2 & -2 & -1
%       \end{bmatrix}$$

%     Test Case Derivation: The stored chemical equation is valid and relatively
%     small, but larger than the trivial one from
%     \testref{test_convert_small_valid}.

%     How test will be performed: The outputted matrix form of the stored
%     chemical equation will be automatically compared to the expected output.

%   \item[T\refstepcounter{testnum}\thetestnum \label{test_convert_large_valid}:]
%     \textbf{Test for Converting Large Valid Chemical Equation}

%     Control: Automated

%     Initial State: \progname{} is started with a representation of the equation
%     $\ce{KMnO4} + \ce{HCl} \rightarrow \ce{MnCl2} + \ce{KCl} + \ce{Cl2} +
%       \ce{H2O}$ \cite{taylor_balancing_2021} stored in memory.

%     Input: --

%     Output: Some matrix with the rows and columns of the following matrix,
%     potentially in a different order:
%     $$\begin{bmatrix}
%         1 & 0 & 0  & -1 & 0  & 0  \\
%         1 & 0 & -1 & 0  & 0  & 0  \\
%         4 & 0 & 0  & 0  & 0  & -1 \\
%         0 & 1 & 0  & 0  & 0  & -2 \\
%         0 & 1 & -2 & -1 & -2 & 0
%       \end{bmatrix}$$

%     Test Case Derivation: The stored chemical equation is
%     valid and larger than the one from \testref{test_convert_valid}.

%     How test will be performed: The outputted matrix form of the stored
%     chemical equation will be automatically compared to the expected output.

%   \item[T\refstepcounter{testnum}\thetestnum \label{test_convert_inf_cons_mass_valid}:]
%     \textbf{Test for Converting Valid Chemical Equation that is Infeasible due
%       to Conservation of Mass Violation}

%     Control: Automated

%     Initial State: \progname{} is started with a representation of the equation
%     $\ce{C2H6} \rightarrow \ce{CO2} + \ce{H2O}$ stored in memory.

%     Input: --

%     Output: Some matrix with the rows and columns of the following matrix,
%     potentially in a different order:
%     $$\begin{bmatrix}
%         2 & -1 & 0  \\
%         6 & 0  & -2 \\
%         0 & -2 & -1
%       \end{bmatrix}$$

%     Test Case Derivation: The stored chemical equation is infeasible since
%     every element does not exist on both sides of the equation, which violates
%     the Law of Conservation of Mass (\tmref{TM_ConsMass} from the SRS
%     \cite{srs}).

%     How test will be performed: The outputted matrix form of the stored
%     chemical equation will be automatically compared to the expected output.

%   \item[T\refstepcounter{testnum}\thetestnum \label{test_convert_inf_over_valid}:]
%     \textbf{Test for Converting Valid Chemical Equation that is Infeasible due
%       to Overconstrained System}

%     Control: Automated

%     Initial State: \progname{} is started with a representation of the equation
%     $\ce{K4FeC6N6} + \ce{K2S2O3} \rightarrow \ce{CO2} + \ce{K2SO4} + \ce{NO2} +
%       \ce{FeS}$ \cite{hamid_balancing_2019} stored in memory.

%     Input: --

%     Output: Some matrix with the rows and columns of the following
%     matrix\footnote{While
%       \cite{hamid_balancing_2019} does not include a matrix representation, the
%       system of equations it provides has a typo: the equation for $\ce{N}$
%       should be $6x_1 = x_5$, since $\ce{NO2}$ only has one $\ce{N}$.},
%     potentially in a different order:
%     $$\begin{bmatrix}
%         4 & 2 & 0  & -2 & 0  & 0  \\
%         1 & 0 & 0  & 0  & 0  & -1 \\
%         6 & 0 & -1 & 0  & 0  & 0  \\
%         6 & 0 & 0  & 0  & -1 & 0  \\
%         0 & 2 & 0  & -1 & 0  & -1 \\
%         0 & 3 & -2 & -4 & -2 & 0
%       \end{bmatrix}$$

%     Test Case Derivation: The stored chemical equation is infeasible since
%     each compound has more than one element, so changing any coefficient
%     affects the number of some other element, causing a chain reaction that
%     does not converge. There is no solution to this system other than the
%     trivial solution ($\mathbf{0}$) \cite{hamid_balancing_2019}.

%     How test will be performed: The outputted matrix form of the stored
%     chemical equation will be automatically compared to the expected output.

%   \item[T\refstepcounter{testnum}\thetestnum \label{test_convert_nonstoich_valid}:]
%     \textbf{Test for Converting Valid Chemical Equation with Nonstoichiometric
%       Compound}

%     Control: Automated

%     Initial State: \progname{} is started with a representation of the equation
%     $\ce{Fe_{0.95}O} + \ce{O2} \rightarrow \ce{Fe2O3}$
%     \cite{doubtnut_when_nodate} stored in memory.

%     Input: --

%     Output: Some matrix with the rows and columns of the following matrix,
%     potentially in a different order:
%     $$\begin{bmatrix}
%         0.95 & 0 & -2 \\
%         1    & 2 & -3
%       \end{bmatrix}$$

%     Test Case Derivation: The stored chemical equation contains a
%     nonstoichiometric compound (i.e., one with a fractional subscript).

%     How test will be performed: The outputted matrix form of the stored
%     chemical equation will be automatically compared to the expected output.

% \end{enumerate}

% \subsubsection{Feasibility Checking Testing}

% To balance a chemical equation, the equation must be able to be balanced. This
% section defines tests for determining if a chemical equation is feasible from
% R3 of the SRS \sjc{Add link}. \sjc{Justify the choice of these specific tests.}

% \begin{enumerate}

%   \item[T\refstepcounter{testnum}\thetestnum \label{test_small_valid_feas}:]
%     \textbf{Test for Feasibility of Small Valid Chemical Equation}

%     Control: Automated

%     Initial State: \progname{} is started.

%     Input: $\begin{bmatrix}
%         2 & -3
%       \end{bmatrix}$.

%     Output: $\textsc{T}$

%     Test Case Derivation: The matrix represents a chemical equation that is
%     valid and trivial.

%     How test will be performed: The outputted Boolean will be automatically
%     compared to the expected output.

%   \item[T\refstepcounter{testnum}\thetestnum \label{test_valid_feas}:]
%     \textbf{Test for Feasibility of Valid Chemical Equation}

%     Control: Automated

%     Initial State: \progname{} is started.

%     Input:
%     $\begin{bmatrix}
%         2 & 0 & -1 & 0  \\
%         6 & 0 & 0  & -2 \\
%         0 & 2 & -2 & -1
%       \end{bmatrix}$ from \cite{hamid_balancing_2019}.

%     Output: $\textsc{T}$

%     Test Case Derivation: The matrix represents a chemical equation that is
%     valid and relatively small, but larger than the trivial one from
%     \testref{test_small_valid_feas}.

%     How test will be performed: The outputted Boolean will be automatically
%     compared to the expected output.

%   \item[T\refstepcounter{testnum}\thetestnum \label{test_large_valid_feas}:]
%     \textbf{Test for Feasibility of Large Valid Chemical Equation}

%     Control: Automated

%     Initial State: \progname{} is started.

%     Input: $\begin{bmatrix}
%         1 & 0 & 0  & -1 & 0  & 0  \\
%         1 & 0 & -1 & 0  & 0  & 0  \\
%         4 & 0 & 0  & 0  & 0  & -1 \\
%         0 & 1 & 0  & 0  & 0  & -2 \\
%         0 & 1 & -2 & -1 & -2 & 0
%       \end{bmatrix}$

%     Output: $\textsc{T}$

%     Test Case Derivation: The matrix represents a chemical equation that is
%     valid and larger than the one from \testref{test_valid_feas}.

%     How test will be performed: The outputted Boolean will be automatically
%     compared to the expected output.

%   \item[T\refstepcounter{testnum}\thetestnum \label{test_cons_mass_valid_inf}:]
%     \textbf{Test for Feasibility of  Valid Chemical Equation that is Infeasible
%       due to Conservation of Mass Violation}

%     Control: Automated

%     Initial State: \progname{} is started.

%     Input: $\begin{bmatrix}
%         2 & -1 & 0  \\
%         6 & 0  & -2 \\
%         0 & -2 & -1
%       \end{bmatrix}$

%     Output: $\textsc{F}$

%     Test Case Derivation: The matrix represents a chemical equation is
%     infeasible since
%     every element does not exist on both sides of the equation, which violates
%     the Law of Conservation of Mass (\tmref{TM_ConsMass} from the SRS
%     \cite{srs}).

%     How test will be performed: The outputted Boolean will be automatically
%     compared to the expected output.

%   \item[T\refstepcounter{testnum}\thetestnum \label{test_over_valid_inf}:]
%     \textbf{Test for Feasibility of Valid Chemical Equation that is Infeasible
%       due to Overconstrained System}

%     Control: Automated

%     Initial State: \progname{} is started.

%     Input:
%     $\begin{bmatrix}
%         4 & 2 & 0  & -2 & 0  & 0  \\
%         1 & 0 & 0  & 0  & 0  & -1 \\
%         6 & 0 & -1 & 0  & 0  & 0  \\
%         6 & 0 & 0  & 0  & -1 & 0  \\
%         0 & 2 & 0  & -1 & 0  & -1 \\
%         0 & 3 & -2 & -4 & -2 & 0
%       \end{bmatrix}$ based on a linear system from
%     \cite{hamid_balancing_2019}\footnote{While \cite{hamid_balancing_2019}
%       does not include a matrix representation, the
%       system of equations it provides has a typo: the equation for $\ce{N}$
%       should be $6x_1 = x_5$, since $\ce{NO2}$ only has one $\ce{N}$.}.

%     Output: $\textsc{F}$

%     Test Case Derivation: The matrix represents a chemical equation that is
%     infeasible since
%     each compound has more than one element, so changing any coefficient
%     affects the number of some other element, causing a chain reaction that
%     does not converge. There is no solution to this system other than the
%     trivial solution ($\mathbf{0}$) \cite{hamid_balancing_2019}.

%     How test will be performed: The outputted Boolean will be automatically
%     compared to the expected output.

%   \item[T\refstepcounter{testnum}\thetestnum \label{test_nonstoich_valid_feas}:]
%     \textbf{Test for Feasibility of Valid Chemical Equation with
%       Nonstoichiometric Compound}

%     Control: Automated

%     Initial State: \progname{} is started.

%     Input:
%     $\begin{bmatrix}
%         0.95 & 0 & -2 \\
%         1    & 2 & -3
%       \end{bmatrix}$

%     Output: $\textsc{T}$

%     Test Case Derivation: The matrix represents a chemical equation that
%     contains a nonstoichiometric compound (i.e., one with a fractional
%     subscript).

%     How test will be performed: The outputted Boolean will be automatically
%     compared to the expected output.

% \end{enumerate}

% \subsubsection{Infeasible Reaction Output Testing}

% If a chemical equation is infeasible, it is important to notify the user as to
% why, in case it is a mistake on the part of the user. This section defines
% tests for outputting a descriptive message if an equation is infeasible from
% R5 of the SRS \sjc{Add link}. \sjc{Justify the choice of these specific tests.}
% The test cases here were chosen to represent different reason that a chemical
% reaction could be infeasible.

% \begin{enumerate}

%   \item[T\refstepcounter{testnum}\thetestnum \label{test_cons_mass_valid_out}:]
%     \textbf{Test for Output for Valid Chemical Equation that is Infeasible
%       due to Conservation of Mass Violation}

%     Control: Manual/Automated \sjc{TBD}

%     Initial State: \progname{} is started, and the following matrix is
%     stored in memory and has been determined to be infeasible:
%     $$\begin{bmatrix}
%         2 & -1 & 0  \\
%         6 & 0  & -2 \\
%         0 & -2 & -1
%       \end{bmatrix}$$

%     Input: --

%     Output: A descriptive message stating that the inputted chemical reaction
%     is infeasible because an element is present on one side of the equation but
%     not the other. \sjc{Is this test meaningful? The output should likely give
%       the specific elements to the user, which isn't possible to ascertain just
%       from the matrix representation.}

%     Test Case Derivation: The matrix represents a chemical equation is
%     infeasible since
%     every element does not exist on both sides of the equation, which violates
%     the Law of Conservation of Mass (\tmref{TM_ConsMass} from the SRS
%     \cite{srs}).

%     How test will be performed: The output will be verified to ensure that an
%     appropriate message is displayed to the user.

%   \item[T\refstepcounter{testnum}\thetestnum \label{test_over_valid_out}:]
%     \textbf{Test for Output for Valid Chemical Equation that is Infeasible
%       due to Overconstrained System}

%     Control: Manual/Automated \sjc{TBD}

%     Initial State: \progname{} is started and the following matrix from
%     \cite{hamid_balancing_2019}\footnote{While \cite{hamid_balancing_2019}
%       does not include a matrix representation, the
%       system of equations it provides has a typo: the equation for $\ce{N}$
%       should be $6x_1 = x_5$, since $\ce{NO2}$ only has one $\ce{N}$.} is
%     stored in memory and has been determined to be infeasible:
%     $$\begin{bmatrix}
%         4 & 2 & 0  & -2 & 0  & 0  \\
%         1 & 0 & 0  & 0  & 0  & -1 \\
%         6 & 0 & -1 & 0  & 0  & 0  \\
%         6 & 0 & 0  & 0  & -1 & 0  \\
%         0 & 2 & 0  & -1 & 0  & -1 \\
%         0 & 3 & -2 & -4 & -2 & 0
%       \end{bmatrix}$$

%     Input: --

%     Output: A descriptive message stating that the inputted chemical reaction
%     is infeasible because the system is overconstrained.

%     Test Case Derivation: The matrix represents a chemical equation that is
%     infeasible since
%     each compound has more than one element, so changing any coefficient
%     affects the number of some other element, causing a chain reaction that
%     does not converge. There is no solution to this system other than the
%     trivial solution ($\mathbf{0}$) \cite{hamid_balancing_2019}.

%     How test will be performed: The output will be verified to ensure that an
%     appropriate message is displayed to the user.

% \end{enumerate}

% \subsubsection{Balancing Testing}

% The main purpose of \progname{} is to balance chemical equations, so this
% functionality must be verified. This section defines tests for balancing a
% chemical equation from
% R4 of the SRS \sjc{Add link}. \sjc{Justify the choice of these specific tests.}
% Note that since swapping rows and/or columns in a matrix doesn't change its
% solution, the matrices outputted may have their rows and/or columns in a
% different order, although swapping columns of the input matrix (e.g., columns
% three and four) means that the associated rows of the output matrix (e.g., rows
% three and four) must also be swapped.

% \begin{enumerate}

%   \item[T\refstepcounter{testnum}\thetestnum \label{test_bal_small_valid}:]
%     \textbf{Test for Balancing Small Valid Chemical Equation}

%     Control: Automated

%     Initial State: \progname{} is started.

%     Input: $\begin{bmatrix}
%         2 & -3
%       \end{bmatrix}$.

%     Output: $\begin{bmatrix}
%         3 \\
%         2
%       \end{bmatrix}$

%     Test Case Derivation: The matrix represents a chemical equation that is
%     valid and trivial.

%     How test will be performed: The outputted matrix will be automatically
%     compared to the expected output.

%   \item[T\refstepcounter{testnum}\thetestnum \label{test_bal_valid}:]
%     \textbf{Test for Balancing Valid Chemical Equation}

%     Control: Automated

%     Initial State: \progname{} is started.

%     Input:
%     $\begin{bmatrix}
%         2 & 0 & -1 & 0  \\
%         6 & 0 & 0  & -2 \\
%         0 & 2 & -2 & -1
%       \end{bmatrix}$ from \cite{hamid_balancing_2019}

%     Output: $\begin{bmatrix}
%         2 \\
%         7 \\
%         4 \\
%         6
%       \end{bmatrix}$ based on \cite{hamid_balancing_2019}

%     Test Case Derivation: The matrix represents a chemical equation that is
%     valid and relatively small, but larger than the trivial one from
%     \testref{test_bal_small_valid}.

%     How test will be performed: The outputted matrix will be automatically
%     compared to the expected output.

%   \item[T\refstepcounter{testnum}\thetestnum \label{test_bal_large_valid}:]
%     \textbf{Test for Balancing Large Valid Chemical Equation}

%     Control: Automated

%     Initial State: \progname{} is started.

%     Input: $\begin{bmatrix}
%         1 & 0 & 0  & -1 & 0  & 0  \\
%         1 & 0 & -1 & 0  & 0  & 0  \\
%         4 & 0 & 0  & 0  & 0  & -1 \\
%         0 & 1 & 0  & 0  & 0  & -2 \\
%         0 & 1 & -2 & -1 & -2 & 0
%       \end{bmatrix}$

%     Output: $\begin{bmatrix}
%         2  \\
%         16 \\
%         2  \\
%         2  \\
%         5  \\
%         8
%       \end{bmatrix}$

%     Test Case Derivation: The matrix represents a chemical equation that is
%     valid and larger than the one from \testref{test_bal_valid}.

%     How test will be performed: The outputted matrix will be automatically
%     compared to the expected output.

%   \item[T\refstepcounter{testnum}\thetestnum \label{test_bal_nonstoich_valid}:]
%     \textbf{Test for Balancing Valid Chemical Equation with
%       Nonstoichiometric Compound}

%     Control: Automated

%     Initial State: \progname{} is started.

%     Input:
%     $\begin{bmatrix}
%         0.95 & 0 & -2 \\
%         1    & 2 & -3
%       \end{bmatrix}$

%     Output:  $\begin{bmatrix}
%         80 \\
%         17 \\
%         38
%       \end{bmatrix}$

%     Test Case Derivation: The matrix represents a chemical equation that
%     contains a nonstoichiometric compound (i.e., one with a fractional
%     subscript).

%     How test will be performed: The outputted matrix will be automatically
%     compared to the expected output.

% \end{enumerate}

% \subsubsection{Feasible Reaction Output Testing}

% Once a chemical reaction is balanced, it must be given to the user for them
% to be able to use it. This section defines tests for outputting a
% representation of the balanced chemical equation from
% R6 of the SRS \sjc{Add link}. \sjc{Justify the choice of these specific tests.}

% \begin{enumerate}

%   \item[T\refstepcounter{testnum}\thetestnum \label{test_small_valid_out}:]
%     \textbf{Test for Output for Small Valid Chemical Equation}

%     Control: Manual/Automated \sjc{TBD}

%     Initial State: \progname{} is started, and some form of the original
%     equation $\ce{O2} \rightarrow \ce{O3}$ and the coefficient matrix
%     $\begin{bmatrix}
%         3 \\
%         2
%       \end{bmatrix}$ are stored in memory.

%     Input: --

%     Output: A representation of the equation $\ce{3O2} \rightarrow \ce{2O3}$
%     \cite[p.~6]{fahey_twenty_2011}.

%     Test Case Derivation: This chemical equation is valid and trivial.

%     How test will be performed: \sjc{TBD}

%   \item[T\refstepcounter{testnum}\thetestnum \label{test_valid_out}:]
%     \textbf{Test for Output for Valid Chemical Equation}

%     Control: Manual/Automated \sjc{TBD}

%     Initial State: \progname{} is started, and some form of the original
%     equation $\ce{C2H6} + \ce{O2} \rightarrow \ce{CO2} + \ce{H2O}$
%     \cite{hamid_balancing_2019} and the following coefficient matrix
%     based on \cite{hamid_balancing_2019} are stored in memory:
%     $$\begin{bmatrix}
%         2 \\
%         7 \\
%         4 \\
%         6
%       \end{bmatrix}$$

%     Input: --

%     Output: A representation of the following equation from
%     \cite[p.~523]{hamid_balancing_2019}:
%     $$\ce{2C2H6} + \ce{7O2} \rightarrow \ce{4CO2} + \ce{6H2O}$$

%     Test Case Derivation: This chemical equation is
%     valid and relatively small, but larger than the trivial one from
%     \testref{test_small_valid_out}.

%     How test will be performed: \sjc{TBD}

%   \item[T\refstepcounter{testnum}\thetestnum \label{test_large_valid_out}:]
%     \textbf{Test for Output for Large Valid Chemical Equation}

%     Control: Manual/Automated \sjc{TBD}

%     Initial State: \progname{} is started, and some form of the original
%     equation $\ce{KMnO4} + \ce{HCl} \rightarrow \ce{MnCl2} + \ce{KCl} +
%       \ce{Cl2} + \ce{H2O}$ \cite{taylor_balancing_2021} and the following
%     coefficient matrix are stored in memory:
%     $$\begin{bmatrix}
%         2  \\
%         16 \\
%         2  \\
%         2  \\
%         5  \\
%         8
%       \end{bmatrix}$$

%     Input: --

%     Output: A representation of the following equation from
%     \cite{taylor_balancing_2021}:
%     $$\ce{2KMnO4} + \ce{16HCl} \rightarrow \ce{2MnCl2} + \ce{2KCl} +
%       \ce{5Cl2} + \ce{8H2O}$$


%     Test Case Derivation: This chemical equation is
%     valid and larger than the one from \testref{test_valid_out}.

%     How test will be performed: \sjc{TBD}

%   \item[T\refstepcounter{testnum}\thetestnum \label{test_nonstoich_valid_out}:]
%     \textbf{Test for Output for Valid Chemical Equation with
%       Nonstoichiometric Compound}

%     Control: Manual/Automated \sjc{TBD}

%     Initial State: \progname{} is started, and some form of the original
%     equation $\ce{Fe_{0.95}O} + \ce{O2} \rightarrow \ce{Fe2O3}$
%     \cite{doubtnut_when_nodate} and the following
%     coefficient matrix are stored in memory:
%     $$\begin{bmatrix}
%         80 \\
%         17 \\
%         38
%       \end{bmatrix}$$

%     Input: --

%     Output: A representation of the following equation:
%     $$\ce{80Fe_{0.95}O} + \ce{17O2} \rightarrow \ce{38Fe2O3}$$

%     Test Case Derivation: This chemical equation
%     contains a nonstoichiometric compound (i.e., one with a fractional
%     subscript).

%     How test will be performed: \sjc{TBD}

% \end{enumerate}

% \subsubsection{Module 1}

% \wss{Include a blurb here to explain why the subsections below cover the module.
%   References to the MIS would be good.  You will want tests from a black box
%   perspective and from a white box perspective.  Explain to the reader how the
%   tests were selected.}

% \begin{enumerate}

%   \item{test-id1\\}

%   Type: \wss{Functional, Dynamic, Manual, Automatic, Static etc. Most will
%     be automatic}

%   Initial State:

%   Input:

%   Output: \wss{The expected result for the given inputs}

%   Test Case Derivation: \wss{Justify the expected value given in the Output field}

%   How test will be performed:

%   \item{test-id2\\}

%   Type: \wss{Functional, Dynamic, Manual, Automatic, Static etc. Most will
%     be automatic}

%   Initial State:

%   Input:

%   Output: \wss{The expected result for the given inputs}

%   Test Case Derivation: \wss{Justify the expected value given in the Output field}

%   How test will be performed:

%   \item{...\\}

% \end{enumerate}

% \subsubsection{Module 2}

% ...

% \subsection{Tests for Nonfunctional Requirements}

% \wss{If there is a module that needs to be independently assessed for
%   performance, those test cases can go here.  In some projects, planning for
%   nonfunctional tests of units will not be that relevant.}

% \wss{These tests may involve collecting performance data from previously
%   mentioned functional tests.}

% \subsubsection{Module ?}

% \begin{enumerate}

%   \item{test-id1\\}

%   Type: \wss{Functional, Dynamic, Manual, Automatic, Static etc. Most will
%     be automatic}

%   Initial State:

%   Input/Condition:

%   Output/Result:

%   How test will be performed:

%   \item{test-id2\\}

%   Type: Functional, Dynamic, Manual, Static etc.

%   Initial State:

%   Input:

%   Output:

%   How test will be performed:

% \end{enumerate}

% \subsubsection{Module ?}

% ...

% \subsection{Traceability Between Test Cases and Modules}

% \wss{Provide evidence that all of the modules have been considered.}

\newpage

\bibliographystyle{ieeetr}

\bibliography{../sources}

\newpage

\section{Appendix}

\subsection{Symbolic Parameters}

Symbolic constants used in test cases are defined here for easy
maintenance.

\begin{itemize}
  \item FIRST\_USE\_TIME = 10 minutes
  \item SUS\_SCORE = 75 (note that ``the average System Usability Scale score
        is 68'' \cite{thomas_how_2015})
  \item USER\_FRAC = 80\%
\end{itemize}

\subsection{Usability Survey} \label{sec_usableSurvey}

This survey (and accompanying methodology) was taken from
\cite{thomas_how_2015}. The user will be asked to
answer the following questions on a scale of one to five, where one means
``Strongly Disagree'' and five means ``Strongly Agree.'' The scores for all
odd-numbered questions will be decremented by one, the even-numbered question
scores will be subtracted from five, and the sum of these new values will be
multiplied by 2.5 to yield the final score.

\begin{enumerate}
  \item I think that I would like to use this system frequently.
  \item I found the system unnecessarily complex.
  \item I thought the system was easy to use.
  \item I think that I would need the support of a technical person to be able
        to use this system.
  \item I found the various functions in this system were well integrated.
  \item I thought there was too much inconsistency in this system.
  \item I would imagine that most people would learn to use this system very
        quickly.
  \item I found the system very cumbersome to use.
  \item I felt very confident using the system.
  \item I needed to learn a lot of things before I could get going with this
        system.
\end{enumerate}

\end{document}