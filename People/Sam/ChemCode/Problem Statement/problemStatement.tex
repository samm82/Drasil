\documentclass{article}

\usepackage{tabularx}
\usepackage{booktabs}
\usepackage{hyperref}
\usepackage{cleveref}
\usepackage{nameref}

\crefformat{footnote}{#2\footnotemark[#1]#3}

\title{Problem Statement and Goals\\\progname}

\author{\authname}

\date{\today}

\input{../Comments}
\input{../Common}

\begin{document}

\maketitle

\begin{table}[hp]
	\caption{Revision History} \label{TblRevisionHistory}
	\begin{tabularx}{\textwidth}{llX}
		\toprule
		\textbf{Date} & \textbf{Developer(s)} & \textbf{Change}              \\
		\midrule
		Jan. 18, 2023 & Sam                   & Create document and fill in
		Problem, Stakeholders, and Goals sections                            \\
		Jan. 19, 2023 & Sam                   & Format for Drasil upload;
		fill in Inputs and Outputs, Environment, and Stretch Goals sections;
		update Stakeholders and Goals sections; and move Environment section \\
		Jan. 20, 2023 & Sam                   & Update Goals and Stretch
		Goals sections, fill in Problem Statement introduction, update
		Drasil reference                                                     \\
		Jan. 22, 2023 & Sam                   & Replace the notion of
		``fractional oxidation state'' with ``nonstoichiometric compound''   \\
		Jan. 31, 2023 & Sam                   & Remove references to SciPy
		and comments about first-person usage and add stretch goals from SRS
		Likely Changes                                                       \\
		Mar. 4, 2023  & Sam                   & Make minor formatting
		improvements                                                         \\
		\bottomrule
	\end{tabularx}
\end{table}

\section{Problem Statement}

The following sections describe the problem statement of \progname{}
by outlining the \nameref{prob}, \nameref{io}, \nameref{env}, and
\nameref{stkhlds}.

\subsection{Problem} \label{prob}

Chemistry is a broad field that studies matter and its interactions
\cite{gordon_chm101_2023}, primarily through chemical reactions.
During a chemical reaction, bonds between some substances break and new ones are
formed to create new substances; these reactions are often written as chemical
equations \cite{lund_introduction_2023}. Despite new chemicals being created,
all atoms from the initial substances, or ``reactants", must be present in the
final substances, or ``products" because of the Law of Conservation of Matter
\cite{lund_introduction_2023}. This means that for a chemical equation to be
useful, it must be balanced by changing the coefficients of the substances
involved in the reaction \cite{lund_introduction_2023}. Additionally, since
molecules only exist in whole numbers (since dividing a molecule changes its
composition into new types of molecules), these coefficients must be whole
numbers, and by convention should be as small as possible
\cite{lund_introduction_2023}.

While these equations can be balanced by hand through the process of
``balancing by inspection" \cite{lund_introduction_2023}, this can be
time-consuming, prone
to error, and inefficient, especially for more complicated chemical reactions.
For each element present in the reaction, an equality can be written for the
number of elements in each substance, with the reactants on one side and the
products on the other, using the coefficients of each substance as the
variables \cite{hamid_balancing_2019}. These equalities then form a system of
linear
equations that can be solved through various methods to yield a relation
between each coefficient, which can then be manipulated to find the required
whole
numbers \cite{lund_introduction_2023, hamid_balancing_2019}. This method can
also identify reactions that are ``infeasible" and balance reactions involving
nonstoichiometric compounds \cite{hamid_balancing_2019}, which are compounds
``in which the numbers of atoms of the elements present cannot be expressed as
a ratio of small whole numbers"
\cite{the_editors_of_encyclopaedia_britannica_nonstoichiometric_2010}.

\subsection{Inputs and Outputs} \label{io}

\noindent Input:

\begin{itemize}
	\item A representation of a chemical equation
\end{itemize}

\noindent Output:

\begin{itemize}
	\item A representation of the inputted chemical equation in its balanced
	      form with the smallest whole number coefficients possible
\end{itemize}

\subsection{Environment} \label{env}

\progname{} will be developed using Drasil
\cite{carette_drasil_2021}, ``a framework for generating
high-quality documentation and code for Scientific Computing Software''
\cite[p. iii]{maclachlan_design_2020} by encapsulating scientific knowledge as
``chunks'' to be reused among projects \cite{maclachlan_design_2020}. By
building this project in Drasil, relevant concepts about chemistry and systems
of linear equations must first be added, along with the capability to solve
these systems. Therefore, a byproduct of this project is that other programs
that use chemistry and/or systems of linear equations can be made using
Drasil. The implementation in Drasil places some constraints on this project.

Since Drasil is built on the idea of reusability, external libraries will be
used to solve these systems of linear equations. This was previously done with
ordinary differential equation (ODE) solvers, since ``creating a complete ODE
solver in Drasil would take considerable time, and there are already many
reliable external libraries \dots tested by long use''
\cite[p. 24]{chen_solving_2022}; these rationales also apply to solvers of
systems of linear equations.

Additionally, Drasil can currently generate code in Python, C++, C\#, Java, and
Swift \cite{chen_solving_2022}. The scope of this project will be limited to
generating code in Python since it is the language in which I
have the most experience.

Since both \progname~and Drasil are purely software systems, the only
hardware involved is the user's computer used to run \progname{}.

\subsection{Stakeholders} \label{stkhlds}
The main stakeholder of this project is Dr.~Spencer Smith, the instructor for
the CAS 741 Development of Scientific Computing Software course for which this
project is being completed. Dr.~Smith and Dr.~Jacques Carette are in charge of
the Drasil project that \progname~seeks to extend, so the implementation and
development process are of significance to them. Likewise, any future
developers of Drasil, including myself, are
potential stakeholders of this project, since they may use features added to
Drasil, such as ideas about chemistry or systems of linear equations. Jason
Balaci, a fellow CAS 741 student and Drasil contributor, is of particular
mention, since
there may be some overlap between our projects so we may be collaborating
throughout this project. I am also a stakeholder of \progname~as the
developer.

More generally, anyone in the field of chemistry in at least a high-school level
may be a stakeholder of this project, as they may use this tool in their work.

\section{Goals}

The goals of this project are to develop a program that\dots

\begin{itemize}
	\item can balance chemical equations (including ones with nonstoichiometric
	      compounds).
	\item can determine if a given chemical reaction is ``infeasible'' (i.e.,
	      not able to be balanced).
	\item is generated by Drasil (along with relevant documentation).
	\item extends Drasil by introducing the concepts from chemistry necessary
	      to balance equations, such as elements, compounds, and reactions.
	\item is written in Python (see \nameref{env}).
	\item uses appropriate external libraries to
	      solve systems of linear equations (see \cite[Ch. 4]{chen_solving_2022}).
\end{itemize}

\newpage

\section{Stretch Goals}

In descending order of priority, the stretch goals of this project are to\dots

\begin{enumerate}
	\item add support for more complex chemical formulas, such as hydrates or
	      those with polymers or isotopes.
	\item add the ability to, given the amount of one substance in a reaction
	      (in moles), calculate the amount of every other substance in the
	      reaction (also in moles).\footnote{\label{chemProbExs}These examples
		      of problems related to chemical equations were taken from
		      \cite{lund_introduction_2023}.}
	\item add the ability to, given the amount of each reactant (in moles),
	      determine the limiting reactant(s) in a reaction.\cref{chemProbExs}
	\item add the ability to, given the amount of each reactant (in moles),
	      determine the theoretical yield of each product and the amount of
	      excess reactant(s).\cref{chemProbExs}
	\item generate code for \progname~in the other languages supported by
	      Drasil. (While using external ODE solvers in Drasil, the developers
	      ``did not find a suitable library for Swift''
	      \cite[p.~24]{chen_solving_2022}; a similar problem may arise when
	      using external system of linear equations solvers,
	      meaning that \progname~may not be generated in all five languages
	      supported by Drasil.)
	\item add the ability to parse valid but incorrectly formatted chemical
	      formulas inputted by the user and format them correctly when
	      outputting them.
	\item do the same as 2-4.~but in terms of mass.\cref{chemProbExs}
	\item add the ability to classify a chemical reaction as ``combination (or
	      synthesis), decomposition, combustion, single replacement, \dots
	      double replacement'' \cite[p.~301]{lund_introduction_2023} or some
	      combination of these.\cref{chemProbExs}
	\item add support for phase labels.\cref{chemProbExs}
	\item add support for precipitation reactions, including solubility and
	      identifying when reactions will not take place.\cref{chemProbExs}
\end{enumerate}

\newpage

\bibliographystyle{ieeetr}
\bibliography{../sources}

\end{document}